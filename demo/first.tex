% 第一个LaTex示例  %注释

\documentclass[UTF8]{ctexart}  % ctexart 表示chinese text article

\title{第一个\LaTeX 示例}  % 文章的标题
\author{ALISURE}  % 文章的作者
\date{\today}  % 写作的日期

\bibliographystyle{plain}  % 声明参考文献的格式

\newtheorem{thm}{定理}  % 定义一个thm环境

\usepackage{graphicx}  % 插图功能的宏包
\usepackage{float}  % 用于图表的不浮动

\usepackage{geometry}  % 设计页面尺寸的宏包
\geometry{a4paper,centering,scale=0.8}  % A4纸张、版心居中、长宽占页面的0.8倍

\usepackage[format=hang,font=small,textfont=it]{caption}  % 设计图表的标题格式

\usepackage{tocbibind}  % 增加目录的项目,默认在目录中加入目录项本身、参考文献索引等项目



% 在begin之前的内容称为导言区(preamble),用来对文档的性质做一些设置,或定义一些命令。

\begin{document}  % 声明了一个document环境,里面是文章的正文部分,也是直接输出的部分。
	
	\maketitle  % 实际输出文章标题
	\tableofcontents  % 目录
	
	\begin{abstract}
		这是一篇对\LaTeX 介绍的文章。
	\end{abstract}
	
	\section{空格的作用}  % 开始新的一节
		这里是一大段文字。首先,我要敲击键盘才能打出文字。然后,我不想敲击键盘打出这些文字。最后,我还是敲出了这些文字。
		你可以看到换行不会分段,只不过这样看起来易读罢了。
		
		你可以看到一个空行可以分段。你也可以看出段前不用打空格就可以有缩进。你可以看出汉字后面的空 格会被忽略,but the space of english   can not be  ignore.
		
	\section{命令与环境}  % 开始新的一节
		这一节处理脚注\footnote{这只是用来演示什么是脚注}和引用内容。
		
		我们来使用一个命令,比如\emph{强调}。我们可以发现,命令都是以反斜线开头的,后接命令名,命令可以带参数和可选参数。
		
		下面是引用的用法:
		\begin{quote}
			这里是引用我的内容吗?引用将环境中的内容单独分行,增加缩进和上下间距,从而突出引用的部分。
		\end{quote}
		
		\begin{thm}[ALISURE]
			啥?这里演示如何使用自定义环境,需要在导言区定义环境。
		\end{thm}
		
	\section{数学公式}
		行内公式(inline formula),又叫正文公式(in-text formula),你可以发现$a+b=c$就是一个行内公式。
		
		对于较长或比较重要的公式,一般单独居中写一行,有时还需要给公式编号,这种叫列表公式(displayed formula):
		\begin{equation}
			a(b+c) = ab + ac
		\end{equation}
		
		数学公式看起来比较复杂,比如一些键盘上没有的字符:
		\begin{equation}
			\angle ACB = \pi / 2
		\end{equation}
		
		再比如有一定结构的公式,如上标、下标、角度:
		\begin{equation}
			A_{b}B^t = B_{b2}C^{t_2} + AC_{b^2}^{2} + 90^\circ
		\end{equation}
		
	\section{图表}
		插入外部图片,需要在导言区使用usepackage命令引入宏包graphicx。
		
		直接插入一张外部图片:\includegraphics[width=3cm,height=3cm]{../img/1.jpg}。插入完毕。
		
		使用单独的环境插入图片:
		\begin{figure}[ht]  % 插入使用的浮动体环境,h表示可以出现在环境周围的文本所在处(here),t表示一页的顶部(top)
			\centering  % 表示后面的内容居中
			\includegraphics[width=3cm,height=3cm,scale=0.1]{../img/1.jpg}  % 插入图片
			\caption{这是一个外部图片。}  % 插图自动编号和标题
			\label{fig:1}  % 给图形定义一个标签,使用该标签可以在文章的其他地方引用caption产生的编号
		\end{figure}
		
		表格一般在\LaTeX 中完成,需要制定表格的行、列对齐模式和表格线。
		
		\begin{tabular}{|rrr|}  % 声明表格的模式,|rrr|表示表格有三列,都是右对齐,在第一列前面和第三列后面各有一条垂直的表格线
			\hline  % 表格中的横线
				序号 & 姓名 & 学号 \\  % 行与行之间用命令 \\ 隔开
			\hline 1 & ALISURE & 1603121715 \\ 2 & 李硕 & 1603121716 \\
			\hline
		\end{tabular}
		
		table环境:
		\begin{table}[H]  % H表示不浮动,可以和正文连在一起。
			\centering  % 表示后面的内容居中
			\begin{tabular}{|rrr|}
				\hline
					序号 & 姓名 & 学号 \\
				\hline
					1 & ALISURE & 1603121715 \\ 2 & 李硕 & 1603121716 \\
				\hline
			\end{tabular}
			\qquad  % 产生2em的空白
				($a^2 + b^2 = c^2$)
		\end{table}
		
	\section{文献}
		首先,需要有.bib格式的文件,可以使用JabRef制作,后面再详细介绍。
		在这里测试引用其他文献\cite{Kline},{}中是在bib中定义的引用标签。
	
	\section{自动化}
		上一章中实际上是使用了bibtex自动化工具,是比较复杂的自动化工具。简单的自动化工具有页码、定理和公式的自动编号等。
		
		目录也是自动从章节命令中提取并写入目录文件中的,tableofcontents即为目录命令。
		
		引用不仅限于参考文献。图表、公式的编号,只要预先设定了标签,也可以通过辅助文件为中介引用。
	
	\nocite{Shiye}  % 在列表中显示并不直接引用的文献
	\bibliography{../bib/math}  % 从文献数据库math中获取文献信息,打印参考文献列表

\end{document}  % 文章的正文结束
